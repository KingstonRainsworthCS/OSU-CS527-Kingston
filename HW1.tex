\documentclass[12pt]{article}
\setlength{\oddsidemargin}{0in}
\setlength{\evensidemargin}{0in}
\setlength{\textwidth}{6.5in}
\setlength{\parindent}{0in}
\setlength{\parskip}{\baselineskip}

\usepackage{amsmath,amsfonts,amssymb}


\begin{document}

CS 527 Winter 2021 \hfill Homework 1\\
Hoang Vu-Kingston

\hrulefill

\begin{enumerate}

	\item\textit{Problem 1}\\
    Case 0 $\rightarrow$ 00000, 1 $\rightarrow$ 11111 and p=0.1\\
    Since majority voting is use for decoding, the probability of error is:
    \begin{align*}
        P_e  &= P_{\text{3/5 bits is wrong}} + P_{\text{4/5 bits is wrong}} + P_{\text{5/5 bits is wrong}} \\
        &= {5 \choose 3}p^3(1-p)^2 + {5 \choose 4}p^4(1-p) + {5 \choose 5}p^5\\
        &= 10 \times 0.1^3 \times 0.9^2 + 5 \times 0.1^4 \times 0.9 + 0.1^5\\
        &= 0.00856
    \end{align*}

    Case 0 $\rightarrow$ 0000000, 1 $\rightarrow$ 1111111 and p=0.1\\
    Similarly:
    \begin{align*}
        P_e  &= P_{\text{4/7 bits is wrong}} + P_{\text{5/7 bits is wrong}} + P_{\text{6/7 bits is wrong}} + P_{\text{7/7 bits is wrong}}\\
        &= {7 \choose 4}p^4(1-p)^3 + {7 \choose 5}p^5(1-p)^2 + {7 \choose 6}p^6(1-p) + {7 \choose 7}p^7\\
        &= 35 \times 0.1^4 \times 0.9^3 + 21 \times 0.1^5 \times 0.9^2 + 7 \times 0.1^6 \times 0.9 + 0.1^7\\
        &= 0.002728
    \end{align*}

    \item\textit{Problem 2}\\
    Possible outputs and their probabilities:
    \begin{align*}
        P(X_1) &= 1/8\\
        P(X_2) & = 7/8 \times 1/7 = 1/8\\
        P(X_3) & = 7/8 \times 6/7 \times 1/6= 1/8\\
        ...
    \end{align*}
    Thus:
    \begin{equation*}
        H(X) = (1/8*log_2(8))*8 = 3
    \end{equation*}

    \item\textit{Problem 3}\\
    Possible outputs and their probabilities:
    \begin{align*}
        P(X_1) &= {5 \choose 1} (3/50) (47/50)^4 = 0.2342...\\
        P(X_2) &= {5 \choose 2} (3/50)^2 (47/50)^3 = 0.0299...\\
        P(X_3) &= {5 \choose 1} (3/50)^3 (47/50)^2 = 0.0019...\\
    \end{align*}
    Thus:
    \begin{align*}
        H(X) &= P(X_1)log_2(1/P(X_1)) + P(X_2)log_2(1/P(X_2)) + P(X_3)log_2(1/P(X_3))\\
             &= 0.6591...
    \end{align*}

    \item\textit{Problem 4}\\
    Scratch that, do normal odd calculation for each case instead (0.5 as the p)
    Since A and B are equally matched, any permutation of outcome is equally likely. The number of permutations for 4 win and 3 lose on either side can be calculate as:
    \begin{equation*}
        2 \times {7 \choose 4} = 70
    \end{equation*}
    Here we assume that the losing team will continue to play and lose for ease of calculation (e.g, AAAA is transform to AAAABBB for permutation calculation purpose)\\
    Thus:
    \begin{equation*}
        H(X) = (1/70*log_2(70))*70 = 6.129...
    \end{equation*}

    Using the same assumption, the possible outputs and probabilities for Y are: \\
    \begin{align*}
        P(Y_1) &= \text{No. of possible 4 games conclusion} \times 1/70 = 2\times {4 \choose 4} \times 1/70 = 1/35\\
        P(Y_2) &= 2 \times {5 \choose 4} - 2 \times {4 \choose 4} \times 1/70 = 8/70\\
        P(Y_3) &= 2 \times {6 \choose 4} - 2 \times {5 \choose 4} \times 1/70 = 2/7\\
        P(Y_3) &= 2 \times {7 \choose 4} - 2 \times {6 \choose 4} \times 1/70 = 4/7\\
    \end{align*}
    Thus:
    \begin{align*}
        H(Y) &= P(Y_1)log_2(1/P(Y_1)) + P(Y_2)log_2(1/P(Y_2)) + P(Y_3)log_2(1/P(Y_3)) + P(Y_4)log_2(1/P(Y_4))\\
             &=
    \end{align*}
\newpage

    \item \textit{Problem 6.3}\\
    The metamorph virus does essentially the same thing after transformation, but it add more line of code to change its identity from scanner and other party
    \item \textit{Problem 6.5}\\
    Since it need a logic bomb and doesn't seem to actively seek out other system while being embed in actual code, it's behaving like a virus that has a payload of system corruption
    \item \textit{Problem 6.6}\\
    Since the code is written inside a legitimate program, it's most likely a built-in backdoor left over from testing phase. If not, it is most likely a virus that aim to gain root access
    \item\textit{Problem 6.10}\\
    If it's an legitimate game, accessing to message service and the address-book are rarely need, if at all. Given that the suspect app is free, it is most likely a spam + trojan that has some feature of a gameplay but main purpose is to use your address book to propagate itself to other system and start malicious action.
    \item\textit{Problem 6.11}\\
    Usually, a pdf file wouldn't need a special permission to interact with already register pdf reader such as adobe or your web browser. The file launcher probably is a malware that will use your permission to do malicious activity. To check your suspicion,use a virus/malware scanner and/or verify with your project manager about the authenticity of the message. The file itself is a typical social engineer-ed malware that target unsuspecting user, and since it did get to you using your manager name, it probably already sent to the entire staff
\newpage


\end{enumerate}

\end{document}
